\documentclass{article}
\usepackage{geometry}
\usepackage{graphicx}
\usepackage{hyperref}
\usepackage{enumitem}
\geometry{margin=1in}
\setlength{\parskip}{1em}

\title{Software Sustainability Audit \\ \large sypec_repo_k3asbwbs}
\author{Sypec Static Auditor}
\date{\today}

\begin{document}
\maketitle

% -----------------------------------------------------------
\section*{Repository}
\begin{itemize}[leftmargin=*]
  \item \textbf{Name}: sypec_repo_k3asbwbs
  \item \textbf{Purpose (inferred)}:\\
        # Developer Guide for Repository `MiguelIbrahimE/echo`

## Overview

This guide provides an overview of the repository `MiguelIbrahimE/echo`, including its structure, expected files, and their purposes. The repository appears to contain various files related to a Node.js application, including TypeScript files, configuration files, and CSS files. However, many files are reported to be corrupted or in binary format, making them unreadable.

## Repository Structure

The repository is organized into several directories, each containing specific types of files. Below is a summary of the expected structure and the purpose of key files:

### Directory Structure

```
/be
  ├── src
  │   ├── auth
  │   │   ├── authRouter.ts
  │   │   ├── loginHandler.ts
  │   ├── routes
  │   │   ├── githubAuthRouter.ts
  │   │   ├── repositoriesRouter.ts
  │   │   ├── userSettingsRouter.ts
  │   ├── services
  │   │   ├── githubCommitService.ts
  │   │   ├── projectStructureExplainer.ts
  │   │   ├── userManualGenerator.ts
  ├── db
  │   ├── init.sql
  ├── .env.example
  ├── package-lock.json
  ├── package.json
  ├── tsconfig.json
  ├── tsconfig.app.json
```

```
/fe
  ├── __tests__
  │   ├── App.test.tsx
  ├── src
  │   ├── SignedUp
  │   │   ├── CSS
  │   │   │   ├── Documents.css
  │   │   │   ├── EditDoc.css
  │   │   │   ├── MyDocuments.css
  │   │   │   ├── navbar.css
  │   │   │   ├── navbar-signedin.css
  │   │   ├── EditDocument.tsx
  │   │   ├── LinkGithubRepo.tsx
  │   │   ├── ProgressModal.tsx
  │   │   ├── Settings.tsx
```

## File Descriptions

### Backend Files (`/be`)

- **`.env.example`**: 
  - Purpose: Template for environment variables.
  - Content: Typically contains key-value pairs for configuration settings.

- **`package.json`**: 
  - Purpose: Manages project dependencies and metadata.
  - Content: Should include project name, version, dependencies, and scripts.

- **`package-lock.json`**: 
  - Purpose: Locks the versions of dependencies for consistent installations.
  - Content: Metadata about installed packages, including version numbers and resolved URLs.

- **`tsconfig.json`**: 
  - Purpose: TypeScript compiler configuration.
  - Content: Should define compiler options, including target and module settings.

- **`init.sql`**: 
  - Purpose: Database initialization script.
  - Content: Should contain SQL commands to set up the database schema.

### Frontend Files (`/fe`)

- **`App.test.tsx`**: 
  - Purpose: Contains tests for the React application.
  - Content: Should include unit tests or integration tests for components.

- **CSS Files**: 
  - Purpose: Stylesheets for the application.
  - Content: Should define styles for various components, but many files appear to be corrupted.

- **TypeScript Files (`.tsx`)**: 
  - Purpose: React components.
  - Content: Should include component definitions, state management, and rendering logic.

## Issues Noted

- **Corrupted Files**: Many files, including TypeScript, CSS, and JSON files, are reported to be in binary format or corrupted, making them unreadable. This includes:
  - TypeScript files in the `/be/src` and `/fe/src/SignedUp` directories.
  - CSS files in the `/fe/src/SignedUp/CSS` directory.
  - Configuration files like `package.json`, `tsconfig.json`, and `package-lock.json`.

## Recommendations

1. **File Integrity Check**: Verify the integrity of the corrupted files. If possible, restore them from a backup or retrieve a valid version from version control.

2. **Regenerate Files**: For files like `package-lock.json`, consider running `npm install` to regenerate the file.

3. **Environment Setup**: Ensure that the environment variables in `.env.example` are correctly set up in a `.env` file for local development.

4. **Testing**: Once files are restored, run tests to ensure that the application functions as expected.

5. **Documentation**: Update documentation to reflect any changes made during the recovery process.

## Conclusion

This guide outlines the structure and purpose of key files in the `MiguelIbrahimE/echo` repository. Due to the presence of corrupted files, further action is required to restore functionality. Follow the recommendations to address these issues and ensure a smooth development experience.
\end{itemize}

% -----------------------------------------------------------
\section*{Scorecard}
\begin{tabular}{rl}
  \textbf{Grade} & A+++\\
  \textbf{Numeric Score} & 95/100\\
  \textbf{Total LOC} & 27434\\
\end{tabular}

% -----------------------------------------------------------
\section*{Energy Model}
\begin{itemize}[leftmargin=*]
    \item 10 users $\rightarrow$ 5.51 kWh / day
    \item 100 users $\rightarrow$ 27.61 kWh / day
    \item 1000 users $\rightarrow$ 138.40 kWh / day
    \item 10000 users $\rightarrow$ 693.65 kWh / day
\end{itemize}


% -----------------------------------------------------------
\section*{Hardware Profile (typical target)}
\begin{itemize}[leftmargin=*]
  \item CPU: AMD EPYC 7763
  \item GPU: NVIDIA A100
  \item RAM: 256 GB
  \item Idle draw: 0.4 kWh / h
\end{itemize}

% -----------------------------------------------------------
\section*{Container Footprint}
\begin{itemize}[leftmargin=*]
  \item Estimated RAM: 256 MB
  \item Estimated Disk: 100 MB
\end{itemize}

% -----------------------------------------------------------
\section*{Key Warnings}
\begin{enumerate}[label=\textbullet,leftmargin=*]
    \item No critical warnings – great job!
\end{enumerate}

% -----------------------------------------------------------
\section*{Security Findings}
\begin{itemize}[leftmargin=*]
    \item No vulnerabilities detected.
\end{itemize}

% -----------------------------------------------------------
\section*{Code Composition}
\begin{itemize}[leftmargin=*]
    \item .yml – 101 LOC
    \item .md – 186 LOC
    \item noext – 166 LOC
    \item .sql – 85 LOC
    \item .json – 15519 LOC
    \item .ts – 1564 LOC
    \item .js – 91 LOC
    \item .html – 14 LOC
    \item .png – 2336 LOC
    \item .cjs – 14 LOC
    \item .example – 19 LOC
    \item .tsx – 1843 LOC
    \item .css – 1139 LOC
    \item .sample – 791 LOC
    \item .idx – 19 LOC
    \item .pack – 3547 LOC
\end{itemize}

% -----------------------------------------------------------
\section*{Test Coverage}
Files with tests: 0 \\
Python files analysed: 0 \\
Estimated coverage: 0.00\%

% -----------------------------------------------------------
\section*{Detected Code Smells}
\begin{itemize}[leftmargin=*]
    \item No obvious smells detected.
\end{itemize}

\end{document}